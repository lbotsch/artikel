% ***********************************************************
% ******************* PHYSICS HEADER ************************
% ***********************************************************
% Version 2
\documentclass[11pt]{article} 
\parindent0pt % No indentation of paragraphs
\parskip11pt
\usepackage{amsmath} % AMS Math Package
\usepackage{amsthm} % Theorem Formatting
\usepackage{amssymb}	% Math symbols such as \mathbb
\usepackage{graphicx} % Allows for eps images
\usepackage{color}
\usepackage{multicol} % Allows for multiple columns
\usepackage[pdfborder={0 0 0}]{hyperref}
%\usepackage{epstopdf}
\usepackage[utf8]{inputenc}
\usepackage[dvips,a4paper,margin=2.5cm,bottom=2cm]{geometry}
\usepackage[T1]{fontenc}
\usepackage{pslatex}
 % Sets margins and page size
\pagestyle{empty} % Removes page numbers
\makeatletter % Need for anything that contains an @ command 
\renewcommand{\maketitle} % Redefine maketitle to conserve space
{ \begingroup 
    \vskip 10pt \begin{center} \huge {\bf \@title} \end{center}
%    \vskip 10pt \begin{center} \normalsize \@author \hskip 20pt \@date \end{center}
    \vskip 30pt
  \endgroup
  \setcounter{footnote}{0}
}
\makeatother % End of region containing @ commands
\renewcommand{\labelenumi}{(\alph{enumi})} % Use letters for enumerate
% \DeclareMathOperator{\Sample}{Sample}
\let\vaccent=\v % rename builtin command \v{} to \vaccent{}
\renewcommand{\v}[1]{\ensuremath{\mathbf{#1}}} % for vectors
\newcommand{\gv}[1]{\ensuremath{\mbox{\boldmath$ #1 $}}} 
% for vectors of Greek letters
\newcommand{\uv}[1]{\ensuremath{\mathbf{\hat{#1}}}} % for unit vector
\newcommand{\abs}[1]{\left| #1 \right|} % for absolute value
\newcommand{\avg}[1]{\left< #1 \right>} % for average
\let\underdot=\d % rename builtin command \d{} to \underdot{}
\renewcommand{\d}[2]{\frac{\mathrm{d} #1}{\mathrm{d} #2}} % for derivatives
\newcommand{\dd}[2]{\frac{\mathrm{d}^2 #1}{\mathrm{d} #2^2}} % for double derivatives
\newcommand{\pd}[2]{\frac{\partial #1}{\partial #2}} 
% for partial derivatives
\newcommand{\pdd}[2]{\frac{\partial^2 #1}{\partial #2^2}} 
% for double partial derivatives
\newcommand{\pdc}[3]{\left( \frac{\partial #1}{\partial #2}
 \right)_{#3}} % for thermodynamic partial derivatives
\newcommand{\ket}[1]{\left| #1 \right>} % for Dirac bras
\newcommand{\bra}[1]{\left< #1 \right|} % for Dirac kets
\newcommand{\braket}[2]{\left< #1 \vphantom{#2} \right|
 \left. #2 \vphantom{#1} \right>} % for Dirac brackets
\newcommand{\matrixel}[3]{\left< #1 \vphantom{#2#3} \right|
 #2 \left| #3 \vphantom{#1#2} \right>} % for Dirac matrix elements
\newcommand{\grad}[1]{\gv{\nabla} #1} % for gradient
\let\divsymb=\div % rename builtin command \div to \divsymb
\renewcommand{\div}[1]{\gv{\nabla} \cdot #1} % for divergence
\newcommand{\curl}[1]{\gv{\nabla} \times #1} % for curl
\let\baraccent=\= % rename builtin command \= to \baraccent
\renewcommand{\=}[1]{\stackrel{#1}{=}} % for putting numbers above =
\newcommand{\integral}[4]{\int_{#1}^{#2} #3 \, \mathrm{d}#4}
\let\oldphi=\phi % rename old phi to oldphi
\renewcommand{\phi}{\varphi}
\newtheorem{prop}{Proposition}
\newtheorem{thm}{Satz}[section]
\newtheorem{lem}[thm]{Lemma}
\theoremstyle{definition}
\newtheorem{dfn}{Definition}
\theoremstyle{remark}
\newtheorem*{rmk}{Bemerkung}
\newtheorem*{beweis}{Beweis}


% Include GnuPlots
\newcommand{\gnuplot}[3]{
    \let\oldincludegraphics=\includegraphics
    \renewcommand{\includegraphics}[1]{\oldincludegraphics{build/#1.pdf}}
    \begin{figure}[htbp]\begin{center}\input{build/#1.tex}\caption{#2}\label{#3}\end{center}\end{figure}
    \let\includegraphics=\oldincludegraphics
}
% ***********************************************************
% ********************** END HEADER *************************
% ***********************************************************
