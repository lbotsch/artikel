%        File: main.tex
%

% ***********************************************************
% ******************* PHYSICS HEADER ************************
% ***********************************************************
% Version 2
\documentclass[11pt]{article} 
\parindent0pt % No indentation of paragraphs
\parskip11pt
\usepackage{amsmath} % AMS Math Package
\usepackage{amsthm} % Theorem Formatting
\usepackage{amssymb}	% Math symbols such as \mathbb
\usepackage{graphicx} % Allows for eps images
\usepackage{color}
\usepackage{multicol} % Allows for multiple columns
\usepackage[pdfborder={0 0 0}]{hyperref}
%\usepackage{epstopdf}
\usepackage[utf8]{inputenc}
\usepackage[dvips,a4paper,margin=2.5cm,bottom=2cm]{geometry}
\usepackage[T1]{fontenc}
\usepackage{pslatex}
 % Sets margins and page size
\pagestyle{empty} % Removes page numbers
\date{}
\author{}
\makeatletter % Need for anything that contains an @ command 
\renewcommand{\maketitle} % Redefine maketitle to conserve space
{ \begingroup 
    \vskip 10pt \begin{center} \huge {\bf \@title} \end{center}
%    \vskip 10pt \begin{center} \normalsize \@author \hskip 20pt \@date \end{center}
    \vskip 30pt
  \endgroup
  \setcounter{footnote}{0}
}
\makeatother % End of region containing @ commands
\renewcommand{\labelenumi}{(\alph{enumi})} % Use letters for enumerate
% \DeclareMathOperator{\Sample}{Sample}
\let\vaccent=\v % rename builtin command \v{} to \vaccent{}
\renewcommand{\v}[1]{\ensuremath{\mathbf{#1}}} % for vectors
\newcommand{\gv}[1]{\ensuremath{\mbox{\boldmath$ #1 $}}} 
% for vectors of Greek letters
\newcommand{\uv}[1]{\ensuremath{\mathbf{\hat{#1}}}} % for unit vector
\newcommand{\abs}[1]{\left| #1 \right|} % for absolute value
\newcommand{\avg}[1]{\left< #1 \right>} % for average
\let\underdot=\d % rename builtin command \d{} to \underdot{}
\renewcommand{\d}[2]{\frac{d #1}{d #2}} % for derivatives
\newcommand{\dd}[2]{\frac{d^2 #1}{d #2^2}} % for double derivatives
\newcommand{\pd}[2]{\frac{\partial #1}{\partial #2}} 
% for partial derivatives
\newcommand{\pdd}[2]{\frac{\partial^2 #1}{\partial #2^2}} 
% for double partial derivatives
\newcommand{\ppd}[3]{\frac{\partial^2 #1}{\partial #2\,\partial #3}}
\newcommand{\pdc}[3]{\left( \frac{\partial #1}{\partial #2}
 \right)_{#3}} % for thermodynamic partial derivatives
\newcommand{\ket}[1]{\left| #1 \right>} % for Dirac bras
\newcommand{\bra}[1]{\left< #1 \right|} % for Dirac kets
\newcommand{\braket}[2]{\left< #1 \vphantom{#2} \right|
 \left. #2 \vphantom{#1} \right>} % for Dirac brackets
\newcommand{\matrixel}[3]{\left< #1 \vphantom{#2#3} \right|
 #2 \left| #3 \vphantom{#1#2} \right>} % for Dirac matrix elements
\newcommand{\grad}[1]{\gv{\nabla} #1} % for gradient
\let\divsymb=\div % rename builtin command \div to \divsymb
\renewcommand{\div}[1]{\gv{\nabla} \cdot #1} % for divergence
\newcommand{\curl}[1]{\gv{\nabla} \times #1} % for curl
\let\baraccent=\= % rename builtin command \= to \baraccent
\renewcommand{\=}[1]{\stackrel{#1}{=}} % for putting numbers above =
\newcommand{\integral}[4]{\int_{#1}^{#2} #3 \, \mathrm{d}#4}
\let\oldphi=\phi % rename old phi to oldphi
\renewcommand{\phi}{\varphi}
\newtheorem{prop}{Proposition}
\newtheorem{thm}{Satz}[section]
\newtheorem{lem}[thm]{Lemma}
\theoremstyle{definition}
\newtheorem{dfn}{Definition}
\theoremstyle{remark}
\newtheorem*{rmk}{Bemerkung}
\theoremstyle{proof}
\newtheorem*{beweis}{Beweis}


% Include GnuPlots
\newcommand{\gnuplot}[3]{
    \let\oldincludegraphics=\includegraphics
    \renewcommand{\includegraphics}[1]{\oldincludegraphics{build/#1.pdf}}
    \begin{figure}[htbp]\begin{center}\input{build/#1.tex}\caption{#2}\label{#3}\end{center}\end{figure}
    \let\includegraphics=\oldincludegraphics
}
% ***********************************************************
% ********************** END HEADER *************************
% ***********************************************************

\title{Variationsprobleme}
\hypersetup
{
    pdfauthor = {Lukas Botsch},
    pdfsubject = {Variationsprobleme},
    pdftitle = {Variationsprobleme},
    pdfkeywords = {DGL, Differentialgleichung, Euler-Lagrange Gleichung, Variationsproblem, Variationsrechnung, Extremalproblem}
}

\begin{document}

\maketitle{}

% Actual content starts here
\section{Welche Art von Problemen versuchen wir hier zu lösen?}
Das sogenannte Variationsproblem (VP)
\begin{equation}
    S[y] := \integral{a}{b}{L(x, y(x), y'(x))}{x} \rightarrow \mathrm{ minimal} \label{eq:vp}
\end{equation}
mit gegebener Funktion $L = L(x, y, p)$ (\textit{Lagrangefunktion}), wobei $y=y(x)$ alle möglichen
Funktionen $y \in C^{1}([a,b])$ mit $y(a) = \alpha$, $y(b) = \beta$ durchläuft,
($\alpha, \beta \in \mathbb{R}$ gegeben).

\subsection*{Beispiel 1}
Problem der kürzesten Entfernung zwischen zwei gegebenen Punkten $(a, \alpha)$ und $(b, \beta)$

\gnuplot{bsp_1}{Wege zwischen zwei Punkten}{graph:bsp_1}

Die Länge einer Kurve von $(a, \alpha)$ nach $(b, \beta)$, gegeben durch $y=y(x), y \in C^1([a,b])$ ist:
\begin{equation}
    \mathcal{L}[y] = \integral{a}{b}{\sqrt{(y(x))^2 + 1}}{x}
\end{equation}

Wir suchen also die Lösung des Variationsproblems (VP) mit $L = \sqrt{1 + p^{2}}$.

\subsection*{Beispiel 2}
Das Hamiltonsche Prinzip der klassischen, nichtrelativistischen Mechanik (HP)

Setze $L = T - U$ (kinetische Energie - potentielle Energie).

Z.B. bei eindimensionaler Bewegung eines Massenpunktes im ortsabhängigen Kraftfeld $f=f(y)$:
\[ L = \frac{m}{2}v^2 - U(y) = \frac{m}{2}(\dot{y})^2 - U(y) , U = - \integral{}{}{f(y)}{y} \]
Unter allen möglichen Wegen des Teilchens von $(t_{0}, \alpha)$ nach $(t_{1}, \beta)$ ist
derjenige Weg $y=y(t)$ gesucht, so dass das \textit{Wirkungsintegral}
\begin{equation}
    S[y] = \integral{t_{0}}{t_{1}}{L}{t} = \integral{t_{0}}{t_{1}}{\left(\frac{m}{2}(\dot{y}(t))^{2} - U(y(t))\right)}{t}
\end{equation}
minimal wird.

Das heißt, die Natur \textit{geizt} mit einer Größe der Dimension \textit{Energie x Zeit}, genauer mit der
physikalischen Größe \textit{Wirkung}.

\section{Lösen des Variationsproblems (VP)}

In \eqref{eq:vp} vermittelt $\mathrm{S}$ eine Abbildung des affinen Unterraums
\[ V := \left\{y\in C^2([a,b]) \mid y(a) = \alpha, y(b) = \beta \right\} \]
nach $\mathbb{R}$, d.h. S ist ein \textit{Funktional} $S: V\rightarrow\mathbb{R}$.

Wir führen nun \eqref{eq:vp} auf die Minimierung einer reellen Funktion zurück. Dazu nehmen wir an:
$y_0\in V$ sei eine Lösung von \eqref{eq:vp}, d.h.
\begin{equation}
    S[y_0] \leq S[y], \forall y\in V
\end{equation}
($y_0$ heißt dann \textit{Minimalfunktion} zum Variationsproblem \eqref{eq:vp}).


Für die Herleitung einer \textit{notwendigen} Bedingung von $y_0$ sei
$\phi\in V_0 = \left\{y\in C^2([a,b]) \mid y(a) = 0, y(b) = 0 \right\}$. Dann ist
$y_\eta = y_0 + \eta\phi, \eta\in(-\eta_0, \eta_0), y_\eta \in V \Rightarrow S[y_0] \leq S[y_\eta] =: h(\eta)$.
Hier ist $h$ eine reelle Funktion $h: (-\eta_0, \eta_0) \rightarrow \mathbb{R}$, mit
\begin{equation}
h(0) \leq h(\eta)
\end{equation}
Daraus folgt, dass $h$ in $\eta = 0$ ein Minimum hat, also $h'(0) = 0$ gilt.

Berechnen wir nun $h'(\eta)$:

\begin{align}
    h'(\eta) = \d{}{\eta} S[y_\eta] &= \d{}{\eta} \left(
        \integral{a}{b}{L(x, y_0(x) + \eta \phi(x), y_0'(x) + \eta \phi'(x))}{x}\right) \notag \\
    &= \integral{a}{b}{ \d{}{\eta}\left(L(x, y_0(x) + \eta \phi(x), y_0'(x) + \eta \phi'(x))\right)}{x}
        \,\,(\mathrm{falls} L\in C^1(\mathbb{R}^3)) \notag \\
    &= \integral{a}{b}{\left( L_y \d{y_\eta}{\eta} + L_p \d{y_\eta'}{\eta}\right)}{x} \notag \\
    &= \begin{aligned}
            \integral{a}{b}{( & L_y(x, y_0(x) + \eta \phi(x), y_0'(x) + \eta \phi'(x)) \phi(x) \\
            & + L_p(x, y_0(x) + \eta \phi(x), y_0'(x) + \eta \phi'(x)) \phi'(x))}{x}
        \end{aligned} \notag \\
    \Rightarrow 0 = h'(0) &= \integral{a}{b}{( L_y(x, y_0(x), y_0'(x)) \phi(x) + L_p(x, y_0(x), y_0'(x)) \phi'(x) )}{x} & \label{eq:erste_var}
\end{align}

Das Integral \eqref{eq:erste_var} in der letzten Zeile nennt man die \textit{erste Variation} von $S[y]$ in Richtung $\phi$ für $y_0$.
(In der Literatur oft mit $\delta S[y_0](\phi)$ abgekürzt)

Damit lautet das Hamilton Prinzip aus Beispiel 2: Die Natur wählt unter allen stetig differenzierbaren Wegen von
$(t_0, \alpha)$ nach $(t_1, \beta)$ denjenigen aus, für den die erste Variation \eqref{eq:erste_var} der Wirkung
$S$ für \textit{beliebige} Funktionen $\phi$ verschwindet.

Formen wir nun den Ausdruck der ersten Variation \eqref{eq:erste_var} mittels partieller Integration um:

\[ h'(0) = \integral{a}{b}{\left( L_y(x, y_0, y_0')\phi - \d{}{x}(L_p(x, y_0, y_0')) \right)}{x} + L_p(x, y_0, y_0')\phi \mid_a^b \]
\begin{equation}
    \Rightarrow 0 = \integral{a}{b}{\left( L_y(x, y_0, y_0') - \d{}{x}(L_p(x, y_0, y_0'))\right) \phi}{x} \,\, , \forall\phi\in V_0
    \label{eq:erste_var_umgef}
\end{equation}

Aus \eqref{eq:erste_var_umgef} folgt, dass $g(x) := L_y(x, y_0, y_0') - \d{}{x}(L_p(x, y_0, y_0')) = 0$ für alle $x\in(a, b)$ und
zwar nach folgendem

\begin{lem}[Fundamentallemma der Variationsrechnung]
    Sei g eine auf $(a, b)$ stetige Funktion und gelte für jedes $\phi\in V_0$ dir Relation
    $\integral{a}{b}{g(x) \phi(x)}{x} = 0$, so folgt $g \equiv 0$ auf $(a, b)$.
\end{lem}

\begin{beweis}
Der Beweis erfolgt indirekt: Wäre $g(x_0) \neq 0$ für $x_0 \in (a, b)$. O.B.d.A sei $g(x_0) > 0$. Dann existiert $\delta > 0$,
so dass $g(x) > 0$ für $x\in (x_0 - \delta, x_0 + \delta)$.\\
Wir wählen nun $\phi_0\in V_0$, so dass:
\begin{equation*}
    \phi_0(x) =
    \begin{cases}
        0 & \text{für $x\in [a, b]\backslash(x_0-\delta, x_0+\delta)$}
        \\
        (x-(x_0-\delta))^4\,(x-(x_0+\delta))^4 & \text{für $x\in(x_0-\delta, x_0+\delta)$}
    \end{cases}
\end{equation*}
Dann ist $\phi_0(x) > 0$ auf $(x_0 - \delta, x_0 + \delta)$ und
\[ \integral{a}{b}{g \phi_0}{x} = \integral{x_0-\delta}{x_0+\delta}{g \phi_0}{x} > 0 \]
Und das steht im Widerspruch zur Voraussetzung! \qed
\end{beweis}

\section{Zusammenfassung}

\begin{thm}
    Sei $L = L(x, y, p)$ zwei mal stetig differenzierbar bezüglich der drei Variablen x, y, p, so muss jede Lösung $y_0\in V$
    des Variationsproblems \eqref{eq:vp} notwendigerweise die \textit{Euler-Lagrange DGL}:
    \begin{equation}
        L_y(x, y_0, y_0') - \d{}{x}\left(L_p(x, y_0, y_0')\right) = 0 \label{eq:elg}
    \end{equation}
    in $(a, b)$ erfüllen, bzw. ausgeschrieben:
    \begin{equation*}
        L_y(x, y_0, y_0') - L_{px}(x, y_0, y_0') - L_{py}(x, y_0, y_0')y_0' - L_{pp}(x, y_0, y_0')y_0'' = 0
    \end{equation*}
\end{thm}

\subsection*{Bemerkungen}
\begin{enumerate}
    \item \eqref{eq:elg} ist gewöhnliche DGL 2. Ordnung für $y_0$. Ihre allgemeine Lösung enthält zwei Konstanten, welche
        sich aus den Randbedingungen $y_0(a) = \alpha$, $y_0(b) = \beta$ ermitteln lassen (Randwert-Problem)
    \item \eqref{eq:elg} ist nur eine notwendige Bedingung für die Minimalfunktion $y_0$. Hinreichende Bedingungen
        erfordern zusätzliche Betrachtung.
\end{enumerate}

\section{Zu den Beispielen}

\subsection*{Beispiel 1}
Die Lagrangefunktion zum Beispiel 1 ist (siehe oben) $L = \sqrt{1 + p^2} \Rightarrow L_y = 0, L_p = \frac{p}{\sqrt{1+p^2}}$.
Die Euler-Lagrange DGL \eqref{eq:elg} lautet also
\begin{eqnarray*}
    L_{pp} y_0'' = 0 &\Leftrightarrow& \frac{y_0''\sqrt{1+y_0'^2} - y_0' \frac{y_0' y_0''}{\sqrt{1+y_0'^2}}}{1+y_0'^2} = 0 \\
        &\Rightarrow& y_0'' (1+(y_0')^2) - (y_0')^2 y_0'' = 0 \\
        &\Rightarrow& y_0'' = 0
\end{eqnarray*}

$y_0$ ist eine lineare Funktion in x: $y_0(x) = \frac{\beta - \alpha}{b - a} (x - a) + \alpha$ \\
Wie erwartet, ist die kürzeste Verbindung zweier Punkte die Gerade durch eben diese Punkte.

\subsection*{Beispiel 2}
Die Lagrangefunktion des Hamilton Prinzips (HP) lautet:
\[ L = \frac{m}{2}p^2 - U(y), L_y = -U' = f, L_p = mp \]
Die Euler-Lagrange DGL \eqref{eq:elg} lautet also für dieses Beispiel
\[ 0 = L_y - \d{}{x}(L_p) = f(y)-\d{}{x}(m\dot{y}) \Rightarrow f(y) = m\ddot{y} \]
Das Hamilton Prinzip führt uns hier auf das bekannte 2. Newtonsche Gesetz.

\section*{Quellen}
\begin{itemize}
    \item Notizen zur Vorlesung \textit{Gewöhnliche Differentialgleichungen} von Prof. Gittel (Uni Leipzig)
    \item H. Fischer, H. Kaul: \textit{Mathematik für Physiker}, Band 3. 2. Auflage. Teubner Verlag, Wiesbaden 2006, ISBN 978-3-8351-0031-2
\end{itemize}

% Actual content ends here
\end{document}
