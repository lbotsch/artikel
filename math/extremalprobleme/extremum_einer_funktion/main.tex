
% File: main.tex
% This is the main file of your article. Begin writing your article after the comment 'Actual content starts here'!


% ***********************************************************
% ******************* PHYSICS HEADER ************************
% ***********************************************************
% Version 2
\documentclass[11pt]{article} 
\parindent0pt % No indentation of paragraphs
\parskip11pt
\usepackage{amsmath} % AMS Math Package
\usepackage{amsthm} % Theorem Formatting
\usepackage{amssymb}	% Math symbols such as \mathbb
\usepackage{graphicx} % Allows for eps images
\usepackage{color}
\usepackage{multicol} % Allows for multiple columns
\usepackage[pdfborder={0 0 0}]{hyperref}
%\usepackage{epstopdf}
\usepackage[utf8]{inputenc}
\usepackage[dvips,a4paper,margin=2.5cm,bottom=2cm]{geometry}
\usepackage[T1]{fontenc}
\usepackage{pslatex}
 % Sets margins and page size
\pagestyle{empty} % Removes page numbers
\date{}
\author{}
\makeatletter % Need for anything that contains an @ command 
\renewcommand{\maketitle} % Redefine maketitle to conserve space
{ \begingroup 
    \vskip 10pt \begin{center} \huge {\bf \@title} \end{center}
%    \vskip 10pt \begin{center} \normalsize \@author \hskip 20pt \@date \end{center}
    \vskip 30pt
  \endgroup
  \setcounter{footnote}{0}
}
\makeatother % End of region containing @ commands
\renewcommand{\labelenumi}{(\alph{enumi})} % Use letters for enumerate
% \DeclareMathOperator{\Sample}{Sample}
\let\vaccent=\v % rename builtin command \v{} to \vaccent{}
\renewcommand{\v}[1]{\ensuremath{\mathbf{#1}}} % for vectors
\newcommand{\gv}[1]{\ensuremath{\mbox{\boldmath$ #1 $}}} 
% for vectors of Greek letters
\newcommand{\uv}[1]{\ensuremath{\mathbf{\hat{#1}}}} % for unit vector
\newcommand{\abs}[1]{\left| #1 \right|} % for absolute value
\newcommand{\avg}[1]{\left< #1 \right>} % for average
\let\underdot=\d % rename builtin command \d{} to \underdot{}
\renewcommand{\d}[2]{\frac{d #1}{d #2}} % for derivatives
\newcommand{\dd}[2]{\frac{d^2 #1}{d #2^2}} % for double derivatives
\newcommand{\pd}[2]{\frac{\partial #1}{\partial #2}} 
% for partial derivatives
\newcommand{\pdd}[2]{\frac{\partial^2 #1}{\partial #2^2}} 
% for double partial derivatives
\newcommand{\ppd}[3]{\frac{\partial^2 #1}{\partial #2\,\partial #3}}
\newcommand{\pdc}[3]{\left( \frac{\partial #1}{\partial #2}
 \right)_{#3}} % for thermodynamic partial derivatives
\newcommand{\ket}[1]{\left| #1 \right>} % for Dirac bras
\newcommand{\bra}[1]{\left< #1 \right|} % for Dirac kets
\newcommand{\braket}[2]{\left< #1 \vphantom{#2} \right|
 \left. #2 \vphantom{#1} \right>} % for Dirac brackets
\newcommand{\matrixel}[3]{\left< #1 \vphantom{#2#3} \right|
 #2 \left| #3 \vphantom{#1#2} \right>} % for Dirac matrix elements
\newcommand{\grad}[1]{\gv{\nabla} #1} % for gradient
\let\divsymb=\div % rename builtin command \div to \divsymb
\renewcommand{\div}[1]{\gv{\nabla} \cdot #1} % for divergence
\newcommand{\curl}[1]{\gv{\nabla} \times #1} % for curl
\let\baraccent=\= % rename builtin command \= to \baraccent
\renewcommand{\=}[1]{\stackrel{#1}{=}} % for putting numbers above =
\newcommand{\integral}[4]{\int_{#1}^{#2} #3 \, \mathrm{d}#4}
\let\oldphi=\phi % rename old phi to oldphi
\renewcommand{\phi}{\varphi}
\newtheorem{prop}{Proposition}
\newtheorem{thm}{Satz}[section]
\newtheorem{lem}[thm]{Lemma}
\theoremstyle{definition}
\newtheorem{dfn}{Definition}
\theoremstyle{remark}
\newtheorem*{rmk}{Bemerkung}
\theoremstyle{proof}
\newtheorem*{beweis}{Beweis}


% Include GnuPlots
\newcommand{\gnuplot}[3]{
    \let\oldincludegraphics=\includegraphics
    \renewcommand{\includegraphics}[1]{\oldincludegraphics{build/#1.pdf}}
    \begin{figure}[htbp]\begin{center}\input{build/#1.tex}\caption{#2}\label{#3}\end{center}\end{figure}
    \let\includegraphics=\oldincludegraphics
}
% ***********************************************************
% ********************** END HEADER *************************
% ***********************************************************

\title{Extrema einer Funktion}
\hypersetup
{
    pdfauthor = {Lukas Botsch},
    pdfsubject = {Extremalstellen und Extrema einer Funktion finden},
    pdftitle = {Extrema einer Funktion},
    pdfkeywords = {Extremum, Funktion, Analysis, Ableitung}
}

\begin{document}
\maketitle{}

% Ein paar nützliche Befehle:
%
% \gnuplot{name}{Titel}{label}  Fügt das gnuplot 'figures/[name].plot' mit dem gegebenen title und label als Figur ein
% 
% \begin{align}
%     % Integral
%     \int_a^b f(x) \,\mathrm{d}x\\
%
%     % Ableitung
%     \d{f}{x}
%
%     % Zweite Ableitung
%     \dd{f}{x}
%
%     % Partielle Ableitung
%     \pd{f}{x}
%
%     % Zweite partielle Ableitung
%     \pdd{f}{x}
%
%     % Nabla notation von Gradient, Divergenz und Krümmung
%     \grad{f} \div{f} \curl{f}
% \end{align}
%
% \begin{thm}[Titel]
%     Ein Satz ...
% \end{thm}
% \begin{beweis}
%     Der Beweis zum Satz ...
%     \qed % setzt ein rechteck ans Ende der Zeile
% \end{beweis}

% Actual content starts here

\section{Welche Art von Problemen versuchen wir hier zu lösen?}

Bei gegebener Funktion $f: U \subset \mathbb{R}^n \rightarrow \mathbb{R}, y = f(x)$ suchen wir Werte $x_{min} \in U$,
so dass für jedes $x$ aus einer Umgebung von $x_{min}$ gilt: $f_{min} := f(x_{min}) \leq f(x)$, bzw. $x_{max} \in U$,
so dass für jedes $x$ aus einer Umgebung von $x_{max}$ gilt: $f_{max} := f(x_{max}) \geq f(x)$. \\
$f_{min}$ nennen wir \textit{relatives Minimum}, $f_{max}$ \textit{relatives Maximum} von f.

Wenn sogar $f_{min} \leq f(x)$ für {\it alle} $x\in U$ gilt, handelt es sich um ein \textit{absolutes} Minimum.\\
Wenn $f_{max} \geq f(x)$ für {\it alle} $x\in U$ gilt, handelt es sich um ein \textit{absolutes} Maximum.

\subsection*{Beispiel}

Die Funktion $f: \mathbb{R} \rightarrow \mathbb{R}, f(x) = x^2$ nimmt an der stelle $x_{min} = 0$ ein absolutes Minimum an.
\gnuplot{bsp_1}{Minimum von $f(x) = x^2$ an der stelle $x = 0$}{fig:bsp_1}

\section{Finden der Extremalstellen und -werte einer Funktion einer Veränderlichen}

Die erste Frage, die wir uns stellen sollten, ist die der Existenz von Extremalstellen. Unter welchen Bedingungen können wir 
Aussagen darüber treffen, ob die Funktion $f(x)$ ein Minimum, bzw. Maximum annimmt?

\subsection{Existenzsatz von Extremalstellen}
\begin{thm}
    Sei $f: I = [a,b] \rightarrow \mathbb{R}$ stetig und sei $I$ ein \textit{kompaktes} Intervall (d.h. {\it abgeschlossen} und
    {\it beschränkt}). Dann ist $f$ in $I$ beschränkt und nimmt ihr Minimum und Maximum an. Dabei können diese auch in den
    Randstellen $a, b$ angenommen werden.
\end{thm}
\begin{beweis}
    Sei $\eta = \mathrm{sup}f(I)$ und $\alpha_n = \eta - \frac{1}{n}$ bzw. $\alpha_n = n$, falls $\eta = \infty$ ist. Für jedes
    $n$ gibt es ein $x_n\in I$ mit $f(x_n) > \alpha_n$. Es strebt dann $f(x_n)\rightarrow\eta$. Die Folge $(x_n)$ besitzt eine
    konvergente Teilfolge; ihr Limes sei $x^\ast$. Die zugehörige Teilfolge von $(f(x_n))$ konvergiert aufgrund der Stetigkeit
    von $f$ gegen $f(x^\ast)$, andererseits gegen $\eta$. Also ist $\eta = f(x^\ast) < \infty$. Beim Infimum wird analog verfahren.
    \qed
\end{beweis}

Wir wollen uns nun Kriterien für solche Extremalstellen überlegen.

\subsection{Bedingung für eine Extremalstelle}

Wenn $f$ an der Stelle $a < x_0 < b$ ein Minimum annimmt, so folgt aus der Definition, dass sich an der Stelle $x_0$ die Richtung
der Kurve von $f$ ändern muss. Es ist nämlich $f(x_0) \leq f(x)$ für alle $x$ aus einer Umgebung von $x_0$. Also muss an dieser
Stelle die Ableitung von $f$ gleich Null sein: $f'(x_0) = 0$. Die selbe Überlegung gilt auch für das Maximum.

Damit haben wir eine notwendige Bedingung für eine Extremalstelle gefunden. Suchen wir also alle Werte $x_i\in I$, für die gilt
$f'(x_i) = 0$. Nun müssen wir nur noch prüfen, ob $f$ an diesen Stellen ein Minimum, bzw. Maximum annimmt.

\subsection{Hinreichende Bedingungen}

Ist $f$ $n+1$-mal differenzierbar, und ist $f^{(2)}(x_0) = f^{(3)}(x_0) = \cdots = f^{(n)}(x_0) = 0$ und $f^{(n+1)}(x_0) \neq 0$,
dann folgt:
\begin{align*}
    & \text{falls $n$ ungerade ist und $f^{(n+1)}(x_0) < 0$,} & \text{f nimmt in $x_0$ ein lokales Maximum an,} \\
    & \text{falls $n$ ungerade ist und $f^{(n+1)}(x_0) > 0$,} & \text{f nimmt in $x_0$ ein lokales Minimum an,} \\
    & \text{falls $n$ gerade ist,} & \text{$f$ nimmt in $x_0$ kein Extremum an.}
\end{align*}
Für $n=1$ ergibt sich folgendes: Ist $f'(x_0) = 0$ und $f''(x_0) > 0$ (bzw. $f''(x_0) < 0$), so nimmt $f$ an der Stelle $x_0$
ein Minimum (bzw. Maximum) an.

Ein weiteres hinreichendes Kriterium folgt aus dem Mittelwertsatz: Seien $\alpha,\beta \in I$ und $\alpha < x_0 < \beta$ und gilt
$f'(x) = 0 \Leftrightarrow x = x_0$ für alle $x\in [\alpha,\beta]$ (also ist die einzige Nullstelle der Ableitung in $[a,b]$ $x_0$),
und gilt $f(\alpha) > f(x_0)$ (bzw. $f(\alpha) < f(x_0)$) sowie $f(\beta) > f(x_0)$ (bzw. $f(\beta) < f(x_0)$), so nimmt $f$ in
$x_0$ ein Minimum (bzw. Maximum) an.

\subsection{Beispiele}

\begin{enumerate}
    \item $f(x) = x^2 + 3$. $f'(x) = 2x = 0 \Leftrightarrow x_0 = 0$. $f''(x) = 2 > 0$. Also ist $f(x_0) = 3$ ein lokales Minimum.
    \item $f(x) = x^4 + 1$. $f'(x) = 4x^3 = 0 \Leftrightarrow x_0 = 0$. $f''(x_0) = f^{(3)}(x_0) = 0$. $f^{(4)}(x_0) = 24 > 0$. Also
        nimmt $f$ an der Stelle $x_0 = 0$ ein lokales Minimum an. Und zwar $f(x_0) = 1$.
\end{enumerate}

\section{Finden der Extremalstellen und -werte einer Funktion mehrerer Veränderlichen}

Analog zum eindimensionalen Fall, suchen wir ein Existenzkriterium und Bedingungen für eine Extremalstelle einer Funktion
$f: U \subset \mathbb{R}^n \rightarrow \mathbb{R}$.

\subsection{Existenzsatz}

Der Existenzsatz aus dem eindimensionalen Fall lässt sich auf den mehrdimensionalen Fall übertragen:
Ist $U$ eine kompakte (d.h. abgeschlossene und beschränkte) Menge und ist $f$ auf $U$ stetig, so nimmt $f$ in $U$
ihr Minimum und Maximum an. Dabei können die Extremalstellen in den Randpunkten von $U$ liegen.

\subsection{Notwendige Bedingung für ein Extremum}

Existiert $f$ in einer Umgebung von $\xi$, existiert $\mathrm{grad} f(\xi)$ und hat $f$ an der Stelle $\xi$ ein lokales
Extremum, so ist $\mathrm{grad} f(\xi) = 0$. Man nennt die Punkte $\xi$ mit $\mathrm{grad} f(\xi) = 0$ {\it stationäre}
oder {\it kritische} Punkte von $f$.

\begin{beweis}
    Da die Funktion $g(t) = f(t, \xi_2, \cdots, \xi_n)$ an der Stelle $t = \xi_1$ ein lokales Extremum hat, ist $g'(\xi_1) = 
    \pd{f}{x_1}(\xi) = 0$ nach dem Kriterium im eindimensionalen Fall. Entsprechend für die anderen Ableitungen. \qed
\end{beweis}

Analog zum eindimensionalen Fall, suchen wir nun noch hinreichende Kriterien, um die Suche nach den Extremalstellen zu vereinfachen.

\subsection{Hinreichende Kriterien für ein Extremum}

Sei $G\subset U$ offen, $f$ zweimal differenzierbar auf $G$ ($f\in C^2(G)$), $\xi \in G$ und $\mathrm{grad} f(\xi) = 0$.
Dann lässt sich die Frage, ob $f$ an der Stelle $\xi$ ein Extremum besitzt, anhand der Hesse-Matrix
\[ H_f(\xi) = \left(\frac{\partial^2 f}{\partial x_i \, \partial x_j}\right)_{i,j = 1, \cdots, n} =
    \left( \begin{array}{cccc}
        \ppd{f}{x_1}{x_1} & \ppd{f}{x_1}{x_2} & \cdots & \ppd{f}{x_1}{x_n} \\
        \ppd{f}{x_2}{x_1} & \ppd{f}{x_2}{x_2} & \cdots & \ppd{f}{x_2}{x_n} \\
        \vdots & \vdots & \ddots & \vdots \\
        \ppd{f}{x_n}{x_1} & \ppd{f}{x_n}{x_2} & \cdots & \ppd{f}{x_n}{x_n}
    \end{array} \right) \]
folgendermaßen beantworten:

\begin{tabular*}{0.75\textwidth}{@{\extracolsep{\fill}}llcl}
    $H_f(\xi)$ & positiv definit & $\Rightarrow$ & lokales Minimum im strengen Sinn, \\
               & negativ definit & $\Rightarrow$ & lokales Maximum im strengen Sinn, \\
               & indefinit       & $\Rightarrow$ & kein Extremum.
\end{tabular*}

Ist $H_f(\xi)$ semi-definit, so müssen weitere Untersuchungen betrieben werden um eine Aussage treffen zu können.

Die Definitheit der Hesse-Matrix ist folgendermaßen definiert:

\begin{tabular*}{0.75\textwidth}{@{\extracolsep{\fill}}lll}
    $H_f$ ist & positiv definit,     & wenn $x^\top H_f x > 0$ für alle $x \neq 0$, \\
              & positiv semidefinit, & wenn $x^\top H_f x \geq 0$ für alle $x$, \\
              & negativ definit,     & wenn $x^\top H_f x < 0$ für alle $x \neq 0$, \\
              & negativ semidefinit, & wenn $x^\top H_f x \leq 0$ für alle $x$, \\
              & indefinit,           & wenn keins der oberen zutrifft.
\end{tabular*}

Im zweidimensionalen Fall $f: G\subset\mathbb{R}^2\rightarrow\mathbb{R}, z=f(x, y)$ lässt sich die Definitheit der Hesse-Matrix
einfacher anhand der Diskriminante $D = f_{xx}f_{yy} - f_{xy}^2$ feststellen:

\begin{tabular*}{0.75\textwidth}{@{\extracolsep{\fill}}lll}
    $D > 0 \Rightarrow H_f$ & positiv definit, & falls $f_{xx} > 0$, \\
                            & negativ definit, & falls $f_{xx} < 0$, \\
    $D = 0 \Rightarrow H_f$ & positiv semidefinit, & falls $f_{xx} > 0$ oder $f_{xx} = 0$ und $f_{yy}\geq0$, \\
                            & negativ semidefinit, & falls $f_{xx} < 0$ oder $f_{xx} = 0$ und $f_{yy}\leq0$, \\
    $D < 0 \Rightarrow H_f$ & indefinit. &
\end{tabular*}

\subsection{Beispiel}

$f(x,y) = e^{xy} + x^2 + 2y^2$. $f_x = ye^{xy} + 2x$, $f_y = xe^{xy} + 4y$, \\
$f_{xx} = y^2e^{xy} + 2$, $x_{yy} = x^2e^{xy} + 4$, $f_{xy} = e^{xy}(1+y^2)$ \\
Die Diskriminante der Hesse-Matrix $D = f_{xx}f_{yy} - f_{xy}^2 = (y^2e^{xy} + 2)(x^2e^{xy} + 4) - e^{xy}(1+y^2)$ \\
Offenbar ist der Punkt $(0,0)$ ein stationärer Punkt und $D(0,0) = 7 > 0$. Also nimmt $f$ in $(0,0)$ ein lokales
Minimum an.

\gnuplot{bsp_2}{$f(x,y) = e^{xy} + x^2 + 2y^2$}{fig:bsp_2}

\section{Extremum mit Nebenbedingungen}

Wir werden uns nun mit dem Extremalproblem im mehrdimensionalen Fall beschäftigen, in dem weitere Bedingungen an die Variablen
$x = (x_1, \cdots, x_n)$ gestellt sind, sogenannte {\it Nebenbedingungen}.

Gegeben seien eine Funktion $f$ und ein m-Tupel von Funktionen $\phi = (\phi_1, \cdots, \phi_m)$ auf $G\subset\mathbb{R}^n, m<n$.
$\mathcal{F}_\phi = \left\{x\in\mathbb{R}^n : \phi_j(x) = 0, j=1,\cdots,m \right\}$.

Wir suchen ein relatives Extremum bezüglich der Nebenbedingung $\phi = 0$.

Die notwendige Bedingung für ein Extremum lautet hier:\\
Ist $x_0$ relatives Extremum für $f$ bezüglich der Nebenbedingung $\phi = 0$, dann existiert $\lambda\in\mathbb{R}^m$ mit
\[ (\mathrm{grad}f)(x_0) = (\mathrm{grad}(\lambda\cdot\phi))(x_0) \]

Setzen wir nun $F = f - \lambda_1\phi_1 - \lambda_2\phi_2 - \cdots - \lambda_m\phi_m, \lambda_1,\cdots,\lambda_m\in\mathbb{R}$.
Dann ist 
\[ \pd{F}{x_j} = \pd{f}{x_j} - \lambda_1\pd{\phi_1}{x_j} - \cdots - \lambda_m\pd{\phi_m}{x_j}
    = \pd{f}{x_j} - \pd{(\lambda_1\phi_1 + \cdots + \lambda_m\phi_m)}{x_j} = \pd{f}{x_j} - \pd{(\lambda\phi)}{x_j} \]
D.h. die Bedingung $\pd{F}{x_j} = 0, j = 1,\cdots,m$ sind äquivalent zu $\mathrm{grad}f = \mathrm{grad}(\lambda\phi)$.

Also haben wir $n$ Gleichungen

\begin{equation}
    \pd{F}{x_j} = 0 \,\, , j = 1, \cdots, n
    \label{eq:nb1}
\end{equation}

und $m$ Gleichungen

\begin{equation}
    \phi_1 = 0, \cdots, \phi_m = 0
    \label{eq:nb2}
\end{equation}

für die $n+m$ Variablen $x_1, \cdots, x_n, \lambda_1, \cdots, \lambda_m$.\\
$\lambda_1, \cdots, \lambda_m$ heißen {\it Lagrange Multiplikatoren}.

\subsection{Beispiel}

$f(x_1, x_2, x_3) = x_1^2 + x_2^2 + x_3^2 = ||x||^2$ \\
$\phi_1(x_1, x_2, x_3) = x_1^2 + x_2^2 - 1$, $\phi_2(x_1, x_2, x_3) = x_1 + x_2 + x_3 - 1$

$\mathrm{grad}f = (2x_1, 2x_2, 2x_3), \mathrm{grad}\phi_1 = (2x_1, 2x_2, 0), \mathrm{grad}\phi_2 = (1, 1, 1)$ \\
Lösen wir das System:

\begin{align*}
    \left. \begin{array}{r}
        2x_1 = \lambda_1 2 x_1 + \lambda_2 \\
        2x_1 = \lambda_1 2 x_1 + \lambda_2 \\
        2x_1 = \lambda_1 2 x_1 + \lambda_2
    \end{array} \right) \leadsto \eqref{eq:nb1}
    \\ \\
    \left. \begin{array}{r}
        x_1^2 + x_2^2 = 1 \\
        x_1 + x_2 + x_3 = 1
    \end{array} \right) \leadsto \eqref{eq:nb2}
\end{align*}

\begin{align}
    \lambda_2 = 2x_3 \notag \\
    \leadsto (1-\lambda_1)x_1 = x_3, (1-\lambda_1)x_2 = x_3 \label{eq:nb3}
\end{align}

\begin{align*}
    \text{\underline{Fall 1:} } & \lambda_1 = 1 \leadsto x_3 = 0 \\
        & x_1 + x_2 = 1, x_2 = 1-x_1 \\
        & x_1^2 + (1-x_1)^2 = 1 \leadsto 2x_1(x_1 - 1) = 0 \\
        & \text{Zwei Lösungen: } x_1 = (1, 0, 0), x_2 = (0, 1, 0) \\
    \text{\underline{Fall 2:} } & \lambda_1 \neq 1 \\
        & \eqref{eq:nb3} \leadsto (1-\lambda_1)(x_1-x_2) = 0, \text{Weil $\lambda_1\neq1$ folgt $x_1 = x_2$} \\
        & x_1^2 + x_2^2 = 2x_1^2 = 1, x_1 = \pm\frac{1}{\sqrt{2}}, x_3 = 1-x_1-x_2 = 1\mp\frac{2}{\sqrt{2}} = 1\mp\sqrt{2} \\
        & \text{Zwei Lösungen: } x_3 = (\frac{1}{\sqrt{2}}, \frac{1}{\sqrt{2}}, 1-\sqrt{2}),
            x_4 = (-\frac{1}{\sqrt{2}}, -\frac{1}{\sqrt{2}}, 1+\sqrt{2}) \\
\end{align*}

Wir haben also vier Punkte, die die notwendige Bedingung für relative Extrema unter Nebenbedingungen erfüllt.
\begin{align*}
    & f(x_1) = f(x_2) = 1 \\
    & f(x_3) = \frac{1}{2} + \frac{1}{2} + (1 - \sqrt{2})^2 = 4 - 2\sqrt{2} \\
    & f(x_4) = \frac{1}{2} + \frac{1}{2} + (1 + \sqrt{2})^2 = 4 + 2\sqrt{2}
\end{align*}

f ist eine stetige Funktion auf $\mathcal{F}_\phi$. $\mathcal{F}_\phi$ ist eine abgeschlossene, beschränkte Menge im $\mathbb{R}^3$,
also ist $\mathcal{F}_\phi$ kompakt. Daher besitzt $f$ auf der Menge $\mathcal{F}_\phi$ ein Maximum und ein Minimum. Diese sind dann
auch relative Maxima und Minima.

\begin{align*}
    & \text{Min}f = \text{Min} \{f(x_1), f(x_2), f(x_3), f(x_4)\} = 1 \\
    & \text{Max}f = \text{Max} \{f(x_1), f(x_2), f(x_3), f(x_4)\} = 4 + 2\sqrt{2}
\end{align*}

\section{Quellen}
\begin{itemize}
    \item W. Walter, {\it Analysis 1}, 7. Auflage, Springer Berlin, ISBN 3-540-20388-5
    \item W. Walter, {\it Analysis 2}, 5. erw. Auflage, Springer Berlin, ISBN 3-540-42953-1
\end{itemize}

% Actual content ends here
\end{document}
